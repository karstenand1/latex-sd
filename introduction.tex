\chapter{Introduction}

\section{Problem Statement}

Currently, there is no way to accurately track a mobile device’s position in a given space without the use of GPS or WiFi.
This means that people who are interested in understanding where they currently are in relationship to other people or landmarks and do
not have access to these technologies are not able to get accurate location information.
While most people with modern smartphones do have nearly unlimited access to GPS and WiFi, this does not necessarily exclude them
from the group of people who are affected by this issue.  In many cases, GPS and WiFi fail to provide the fine grain accuracy required
to provide useful information.  Within a building GPS and WiFi can be unreliable because their physical signature is fundamentally not
designed to address the need for indoor or finely tuned positioning. This can cause the distance and elevation that GPS reports back to
lose accuracy. It cannot determine what floor you’re on so if you’re on the second floor or the ninth floor, you would show up in the
same location. In terms of distance, it could show you at least several meters off from your real world location.
Best case scenario, GPS will get someone to the door of a building they are looking for, but after that the person is on their own once
inside of that building. Due to their design, GPS and WiFi cannot help you determine your location once inside a building in many cases,
especially if the building has multiple floors. Often times stores and campuses provide physical maps for visitors. For a physical map
however, there is no map for finding the map. To use the map, a person physically has to move to it to view where they are. Also, if
there is a single map for all visitors, then a person can’t carry it with them once they head for their destination. This means people
can get lost on the way to their destination after leaving the map. Other current solutions try to solve this issue with proximity
detection technology. The problem here is that proximity based systems can only tell a person how close they are relative to something;
it doesn’t give information on the direction or elevation of whatever the person may be looking for. This can cause people to waste time
figuring out what direction they need to head in or even put them a floor above or below their destination.

\section{Objectives}

Our solution
is a software layer that utilizes existing bluetooth hardware to create a highly accurate relative positioning system.
The software layer utilizes a variety of algorithms to interpret the relative signal strength of known bluetooth transmitters
and will be able to return a useful, three dimensional, representation of where a smartphone or other bluetooth device is.
The ability to understand the location of a device in two or three dimensions as opposed to the proximity of a device in one
dimension allows for an exponential increase in the possible applications for such a positioning framework.  Additionally,
the ability to navigate using a smartphone indoors vastly improves the currently nondigital solution of central physical maps
that are currently in use in many large buildings.  The proposed framework offers a software solution that will be cheap and
highly distributable due to the fact that is utilizes existing technologies.  This means the framework will be able to support a
wide variety of applications without needing to wait around for the next breakthrough spectrum standard in positioning technology.

\subsection{Functional Requirements}
Below are the initial requirements that were decided when the project was proposed.
\newline

Critical:
\begin{itemize}
\item The system will return a location when 3+ beacons are available
\item The system will be able to interpret BLE beacon noise levels to produce a precise location
\item The system will give more accurate location data than GPS or WiFi in specific situations
\end{itemize}
Recommended:
\begin{itemize}
\item Offline installation of system does not undermine its usability for devs and admins
\end{itemize}

\subsection{Non-Functional Requirements}

Critical:
\begin{itemize}
\item The framework is easy to use for developers
\item The framework will return location coordinates with an security between 0m and 5m
\item The system will scale to large applications
\end{itemize}
Recommended:
\begin{itemize}
\item Admin interface is easy to use
\item Framework is optimized for battery conservations
\end{itemize}

\subsection{Design Constraints}

\begin{itemize}
\item System must work on iOS
\item Must use low cost BLE beacons
\item System conforms to iBeacon protocol
\end{itemize}
