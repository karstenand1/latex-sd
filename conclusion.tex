\chapter{Conclusion}
\section{Summary}

People who are interested in understanding where they currently are in relationship to other people or landmarks and do
not have access to GPS or Wifi are not able to get accurate location information. In many cases, GPS and WiFi fail to provide the
accuracy required to provide useful information as well.
\newline \par
Our solution is a software layer that utilizes existing bluetooth
hardware to create an accurate relative positioning system. The software layer utilized a variety of algorithms to interpret the
relative signal strength of known bluetooth transmitters and try to return a useful, three dimensional, representation of
where an iOS device is. The framework offered a software solution that is cheap and
highly distributable due to the fact that it utilizes existing technologies.

\section{Lessons Learned}

\section{Future Improvements}

This section sets out to look toward the future and understand how our solution fits in.

\subsection{Scale Matters}
With Bluetooth Low Energy Beacons, there is no one size fits all solution for filling a building. Buildings with more open floor plans will more easily accommodate
these beacons and our algorithms because the signal will propagate off walls less. In a narrow room, a lot of noice will be created making it much harder for the real signal to be distinguished from the rest.
Another problem is how easily the signal is absorbed by the human body. The system can return a different value for location if there are too many people between the iBeacons and the device. Finally, there is no definite number for the amount of iBeacons
needed to fill a floor plan. Although there is the general rule of more being better, there is no definite answer to how many beacons the system really needs.

\subsection{Do Not Force It}
Although Bluetooth Low Energy beacons have been shown to be able to return a position for an indoor positioning system, it does not do it well enough
to be a complete solution. Instead of forcing iBeacons to be the answer and using the software side to correct their shortcomings, a new hardware solution
would be a much better answer to this problem. The hardware could use a better frequency that allows for better frequency analysis but still possibly use
a similar protocol to the iBeacon protocol to allow for ease of use. Instead of trying to force hardware to work with software solutions, this problem should be conquered
by a hardware solution.
