\chapter{Testing and Results}
\begin{itemize}
\item
Unit Testing: Most testing will be done by our developers using manual white box testing. Automated testing will not be used because it would not give us the desired results and will take as much time setting up as it would do to just test manually.

	We will start off with General Feature Testing wherein we will code a particular feature and test it for it's functionality. As the Unit Testing model recommends, every time we write a new feature, we will test it for its functionality ensuring minimization of errors.
    This method will help us break our system into smaller, more manageable parts, the testing of which will be relatively simpler. It will also help us make these small parts ready to be used as a part of the bigger system.
\end{itemize}

\begin{itemize}
\item Regression Testing: As we will progress with the development of the system, we will begin implementing Regression Tests on top of the general feature testing. Unlike what we will do in Unit Testing, in Regression Testing we will write and test a particular feature and test the system. Once everything works fine, we will add the next feature and then retest the system from the beginning, instead of just testing the new feature.

Simply put, regression testing is testing the system in its entirety every time a change or addition is made to it to make sure the already implemented features are not be affected by the new features in an undesirable way.

Regression testing will help us efficiently manage our system whenever a change will be made.
\end{itemize}
\begin{itemize}
\item Ad hoc Testing: Ad hoc testing is performed without a plan of action. It will be performed by improvisation where the testers will find bugs by any means that seemed appropriate. Ad hoc tests will help us spot out any strange bugs that could arise in the operation of the system.
\end{itemize}
